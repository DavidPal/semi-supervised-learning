\section{Preliminaries}
\label{section:preliminaries}

Let $\X$ be a non-empty set. We denote by $\{0,1\}^\X$ the class of all
functions from $\X$ to $\{0,1\}$. A \emph{concept class over a domain $\X$} is a
subset $C \subseteq \{0,1\}^\X$. A \emph{labeled example} is a pair $(x,y) \in
\X \times \{0,1\}$.

A \emph{distribution-independent learning algorithm} is a function
$A:\bigcup_{m=0}^\infty \left(\X \times \{0,1\} \right)^m \to \{0,1\}^\X$. In
other words, the algorithm gets as input a sequence of labeled examples $(x_1,
y_1), (x_2, y_2), \dots, (x_m, y_m)$ and outputs a function from $\X$ to
$\{0,1\}$. We allow the algorithm to output a function that does not belong to
$C$, i.e., the algorithm can be improper. A \emph{distribution-dependent
algorithm} is a function that maps any probability distribution over $\X$ to a
distribution-independent algorithm.

Let $P$ be a probability distribution over a domain $\X$. For any two functions
$f:\X \to \{0,1\}$, $g:\X \to \{0,1\}$ we define the disagreement pseudo-metric
$$
d_P(f,g) = \Pr_{X \sim P}[f(X) \neq g(X)] \; .
$$
Let $C$ be a concept class over $\X$, let $c \in C$, let $\epsilon, \delta \in (0,1)$.
Let  $X_1, X_2, \dots, X_m$ be an i.i.d. sample from $P$. We define the corresponding
labeled sample $T = ((X_1, c(X_1)), (X_2, c(X_2)), \dots, (X_m, c(X_m)))$.
We say that an algorithm $A$, \emph{$\epsilon$-learns} target $c$ from $m$ samples
with probability at least $1 - \delta$ if
$$
\Pr \left[d_P(c,A(T)) \le \epsilon \right]  \ge 1 - \delta \; .
$$
The smallest non-negative integer $m$ such that for any target $c \in C$,
the algorithm $A$, $\epsilon$-learns the target $c$ from $m$ samples
with probability at least $1-\delta$ is denoted by $m(A,C,P,\epsilon,\delta)$.

We recall the standard definitions from learning theory. For any concept $c:\X
\to \{0,1\}$ and any $S \subseteq \X$ we define $\pi(c,S) = \{x \in S ~:~ c(x) =
1 \}$. In other words, $\pi(c,S)$ is the set of examples in $S$ which $c$ labels
$1$. A set $S \subseteq \X$ is \emph{shattered} by a concept class $C$ if for
any subset $S' \subseteq S$ there exists a classifier $c \in C$ such that
$\pi(c,S) = S'$. \emph{Vapnik-Chervonenkis dimension} of a concept class $C$ is
the size of the largest set $S \subseteq \X$ shattered by $C$. A subset $C'$ of
a concept class $C$ is an \emph{$\epsilon$-cover} of $C$ for a probability
distribution $P$ if for any $c \in C$ there exists $c' \in C'$ such that
$d_P(c,c') \le \epsilon$.

To prove our lower bounds we need three general probabilistic results. The first
one is the standard Hoeffding bound. The other two are simple and intuitive
propositions. The first proposition says that if average error $d_P(c,A(T))$ is
high, the algorithm fails to $\epsilon$-learn with high probability. The second
proposition says that the best algorithm for predicting a bit based on some side
information, is to compute conditional expectation of the bit and thresholds it
at $1/2$.

\begin{theorem}[Hoeffding bound]
Let $X_1, X_2, \dots, X_n$ be i.i.d. random variables that lie in interval
$[a,b]$ with probability one and let $p=\frac{1}{n}\sum_{i=1}^n \Exp[X_i]$.
Then, for any $t \ge 0$,
\begin{align*}
\Pr \left[{\frac {1}{n}} \sum_{i=1}^n X_i \ge p + t \right] \le e^{ - 2n t^2/(a-b)^2} \; , \\
\Pr \left[{\frac {1}{n}} \sum_{i=1}^n X_i \le p - t \right] \le e^{ - 2n t^2/(a-b)^2}  \; .
\end{align*}
\end{theorem}

\begin{proposition}[Error probability vs. Expected error]
\label{proposition:error-probability-vs-expected-error}
Let $Z$ be a random variable such that $Z \le 1$ with probability one.
Then,
$$
\Pr[Z > t] \ge \frac{\Exp[Z] - t}{1 - t} \qquad \text{for any $t \in [0, 1)$.}
$$
\end{proposition}

\begin{proof}
We have
\begin{align*}
\Exp[Z]
& \le t \cdot \Pr[Z \le t] + 1 \cdot \Pr[Z > t] \\
& = t \cdot (1 - \Pr[Z > t]) + \Pr[Z > t] \; .
\end{align*}
Solving for $\Pr[Z > t]$ finishes the proof.
\end{proof}

\begin{proposition}[Predicting Single Bit]
\label{proposition:single-bit}
Let $\U$ be a finite non-empty set. Let $U,V$ be random variables (possibly
correlated) such that $U \in \U$ and $V \in \{0,1\}$ with probability one. Let
$f:\U \to \{0,1\}$ be a predictor. Then,
\begin{align*}
\Pr\left[ f(U) \neq V \right]
\ge &\sum_{u \in \U} \left( \frac{1}{2} - \left| \frac{1}{2} -  \Exp \left[V \, \middle| \, U = u\right] \right| \right) 
\\&\hspace{3.5cm}\cdot \Pr[U = u] \; .
\end{align*}
\end{proposition}

\begin{proof}
We have
\begin{align*}
\Pr \left[ f(U) \neq V \right] =& \sum_{u \in \U} \Pr \left[ f(U) \neq V \, \middle| \, U = u \right] 
\\&\hspace{3.5cm}\cdot \Pr[U = u] \; .
\end{align*}
It remains to show that
$$
\Pr\left[ f(U) \neq V \, \middle| \, U = u \right]
\ge
\frac{1}{2} - \left| \frac{1}{2} -  \Exp \left[V \, \middle| \, U = u \right] \right| \; .
$$
Since if  $U=u$, the value $f(U) = f(u)$ is fixed, and hence
\begin{align*}
& \Pr\left[ f(U) \neq V \, \middle| \, U = u \right] \\
& \qquad \ge \min\left\{ \Pr \left[ V = 1 \, \middle| \, U = u \right], \ \Pr \left[ V = 0 \, \middle| \, U = u \right] \right\} \\
& \qquad = \min\left\{ \Exp \left[ V  \, \middle| \, U = u \right], \ 1 - \Exp \left[ V \, \middle| \, U = u \right] \right\} \\
& \qquad = \frac{1}{2} - \left| \frac{1}{2} -  \Exp \left[ V  \, \middle| \, U = u \right] \right|
\end{align*}
We used the fact that $\min\{x, 1 - x\} = \frac{1}{2} - \left| \frac{1}{2} - x \right|$ for all $x \in \R$
which can be easily verified by considering two cases: $x \ge \frac{1}{2}$ and $x < \frac{1}{2}$.
\end{proof}
