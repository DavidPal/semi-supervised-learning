\section{Size of $\epsilon$-cover}
\label{section:epsilon-cover}

In this section, we present $(4e/\epsilon)^{d/(1 - 1/e)}$ upper bound on the
size of the $\epsilon$-cover of any concept class of Vapnik-Chervonenkis
dimension $d$. To prove our result, we need Sauer's lemma. Its proof can be
found, for example, in~\citet[Chapter 3]{Anthony-Bartlett-1999}.

\begin{lemma}[Sauer's lemma]
Let $\X$ be a non-empty domain and let $C \subseteq \{0,1\}^\X$ be a concept class
with Vapnik-Chervonenkis dimension $d$. Then, for any $S \subseteq \X$,
$$
\left| \left\{ \pi(c, S) ~:~ c \in C \right\} \right| \le \sum_{i=0}^d \binom{|S|}{i} \; .
$$
\end{lemma}

We remark that if $n \ge d \ge 1$ then
\begin{equation}
\label{equation:sauer-lemma-estimate}
\sum_{i=0}^d \binom{n}{i} \le \left( \frac{ne}{d} \right)^d
\end{equation}
where $e = 2.71828 \dots$ is the base of the natural logarithm. This follows
from the following calculation
\begin{align*}
\left( \frac{d}{n} \right)^d \cdot \sum_{i=0}^d \binom{n}{i}
& \le \sum_{i=0}^d \binom{n}{i} \left( \frac{d}{n} \right)^i \\
& \le \sum_{i=0}^n \binom{n}{i} \left( \frac{d}{n} \right)^i \\
& = \left(1 + \frac{d}{n} \right)^n \le e^d
\end{align*}
where we used in the last step that $1 + x \le e^x$ for any $x \in \R$.

\begin{theorem}[Size of $\epsilon$-cover]
Let $\X$ be a non-empty domain and let $C \subseteq \{0,1\}^\X$ be a concept
class with Vapnik-Chervonenkis dimension $d$. Let $P$ be any distribution over
$\X$. For any $\epsilon \in (0,1]$, there exists a set $C' \subseteq C$ such that
\begin{equation}
\label{equation:theorem-epsilon-cover}
|C'| \le \left( \frac{4e}{\epsilon} \right)^{d/(1-1/e)}
\end{equation}
and for any $c \in C$ there exists $c' \in C'$ such that $d_P(c,c') \le \epsilon$.
\end{theorem}

\begin{proof}
We say that a set $B \subseteq C$ is an \emph{$\epsilon$-packing} if
$$
\forall c,c' \in B \qquad \qquad c \neq c' \quad \Longrightarrow \quad d_P(c,c') > \epsilon
$$
We claim that there exists a maximal $\epsilon$-packing. In order to show that a
maximal set exists we to appeal to Zorn's lemma. Consider the collection of all
$\epsilon$-packings. We impose partial order on them by set inclusion. Notice
that any totally ordered collection $\{ B_i ~:~ i \in I \}$ of
$\epsilon$-packings has an upper bound $\bigcup_{i \in I} B_i$ that is an
$\epsilon$-packing. Indeed, if $c,c' \in \bigcup_{i \in I} B_i$ such that $c
\neq c'$ then there exists $i \in I$ such that $c,c' \in B_i$ since $\{ B_i ~:~
i \in I \}$ is totally ordered. Since $B_i$ is an $\epsilon$-packing, $d_P(c,c') >
\epsilon$. We conclude that $\bigcup_{i \in I} B_i$ is an $\epsilon$-packing. By
Zorn's lemma, there exists a maximal $\epsilon$-packing.

Let $C'$ be a maximal $\epsilon$-packing. We claim that $C'$ is also an
$\epsilon$-cover of $C$. Indeed, for any $c \in C$ there exists $c' \in C'$ such
that $d_P(c,c') \le \epsilon$ since otherwise $C' \cup \{c\}$ would be an
$\epsilon$-packing, which would contradict maximality of $C'$.

It remains to upper bound $|C'|$. Consider any finite subset $C'' \subseteq C'$.
It suffices to show an upper bound on $|C''|$ and since $C''$ is arbitrary, the
same upper bound holds for $|C'|$. Let $M = |C''|$ and let $c_1, c_2, \dots,
c_M$ be concepts in $C''$. For any $i,j \in \{1,2,\dots,M\}$, $i < j$, let
$$
A_{i,j} = \{ x \in \X ~:~ c_i(x) \neq c_j(x) \} \; .
$$
Let $X_1, X_2, \dots, X_K$ be an i.i.d. sample from $P$. We will choose $K$ later.
Since $d_P(c_i, c_j) > \epsilon$,
$$
\Pr[X_k \in A_{i,j}] > \epsilon \qquad \qquad \text{for $k=1,2,\dots,K$}.
$$
Since there are $\binom{M}{2}$ subsets $A_{i,j}$, we have
\begin{align*}
& \Pr\left[\forall i,j, i < j, \ \exists k, \ X_k \in A_{i,j} \right] \\
& \qquad = 1 - \Pr\left[\exists i,j, i < j, \ \forall k, \ X_k \not \in A_{i,j} \right] \\
& \qquad \ge 1 - \sum_{1 \le i < j \le M} \Pr\left[\forall k, \ X_k \not \in A_{i,j} \right] \\
& \qquad \ge 1 - \sum_{1 \le i < j \le M} (1 - \epsilon)^K \\
& \qquad = 1 - \binom{M}{2} (1 - \epsilon)^K \; .
\end{align*}
For $K = \left\lceil \frac{\ln \binom{M}{2}}{\epsilon} \right\rceil +
1$, the above probability is strictly positive. This means there exists a set $S =
\{x_1, x_2, \dots, x_K\} \subseteq X$ such that $A_{i,j} \cap S$ is non-empty
for every $i < j$. This means that for every for every $i \neq j$, $c_i(S) \neq
c_j(S)$ and hence $M = |C''| = \left| \left\{ \pi(c, S) ~:~ c \in C \right\} \right|$.
Thus by Sauer's lemma
$$
M \le \sum_{i=0}^d \binom{K}{i} \; .
$$
We now show that this inequality implies that $M \le (4e/\epsilon)^{d/(1-1/e)}$. We consider several cases.

Case 1: $d = -\infty$. That is, no set is shattered, and $C = \emptyset$.
Then, $M = 0$ and inequality trivially follows.

Case 2: $d = 0$. Then, $M \le 1$ and the inequality trivially follows.

Case 3a: $d \ge 1$ and $M \le e^d$. Clearly, $M \le e^d \le (4e/\epsilon)^{d/(1 - 1/e)}$.

Case 3b: $d \ge 1$ and $M > e^d$. Then, $K \ge \ln M \ge d$ and
hence by \eqref{equation:sauer-lemma-estimate},
\begin{align*}
M
\le \sum_{i=0}^d \binom{K}{i}
\le \left( \frac{Ke}{d} \right)^d \; .
\end{align*}
Thus,
\begin{align*}
\ln M
& \le d \ln \left( \frac{Ke}{d} \right) \\
& \le d \ln \left( \frac{e \left(\left\lceil \frac{\ln \binom{M}{2}}{\epsilon} \right\rceil + 1 \right)}{d} \right) \\
& \le d \ln \left( \frac{e \left(\frac{\ln \binom{M}{2}}{\epsilon} + 2 \right)}{d} \right) \\
& \le d \ln \left( \frac{e \left(\frac{\ln \binom{M}{2} + 2}{\epsilon} \right)}{d} \right) \\
& \le d \ln \left( \frac{e \left(\frac{2\ln M + 2}{\epsilon} \right)}{d} \right) \\
& \le d \ln \left( \frac{e \left(\frac{4\ln M}{\epsilon} \right)}{d} \right) \\
& = d \left[ \ln \left( \frac{4e}{\epsilon} \right)  + \ln \left(\frac{\ln M}{d} \right) \right] \\
& \le d \ln \left( \frac{4e}{\epsilon} \right) + \frac{1}{e} \ln M \; .
\end{align*}
where in the last step we used that $\ln x \le x/e$ for any $x > 0$.
Hence,
$$
(1 - 1/e) \ln M \le d \ln \left( \frac{4e}{\epsilon} \right)
$$
which implies the lemma.
\end{proof}
