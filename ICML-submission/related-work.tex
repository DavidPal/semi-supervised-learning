\section{Related Work}
\label{section:related-work}

The question of whether knowledge of unlabeled data distribution helps was
proposed and initially studied by \citet{Ben-David-Lu-Pal-2008}; see also
\citet{Lu-2009}. However, they considered only classes with Vapnik-Chervonenkis
dimension at most $1$, or classes with Vapnik-Chervonenkis dimension $d$ but
only distributions for which the size of the $\epsilon$-cover is
$\Theta(1/\epsilon)^{\Theta(d)}$, i.e. the $\epsilon$-cover is as large as it
can be.\footnote{For any concept class with Vapnik-Chervonenkis dimension $d$
and any distribution, the size of the smallest $\epsilon$-cover is at most
$O(1/\epsilon)^{O(d)}$. In fact, as we show in \autoref{section:epsilon-cover},
the size of the smallest $\epsilon$-cover is at most $(e/\epsilon)^d$.} In these
settings, for constant $\epsilon$ and $\delta$, the separation of labeled sample
complexities is at most a constant factor, which is exactly what
\citet{Ben-David-Lu-Pal-2008} proved. In these settings, unlabeled data are
indeed useless. However, these results say nothing about distributions with
$\epsilon$-cover of small size and it ignores the dependency on the
Vapnik-Chervonenkis dimension.

The question was studied in earnest by \citet{Darnstadt-Simon-Szorenyi-2013} who
showed two major results. First, they show that for any non-trivial concept
class $C$ and for every distribution, the ratio of the labeled sample
complexities between distribution-independent and distribution-dependent
algorithms is bounded by the Vapnik-Chervonenkis dimension. Second, they show
that for the class of projections over $\{0,1\}^n$, there are distributions
where the ratio grows to infinity as a function of $n$.

In learning theory, the disagreement metric and $\epsilon$-cover were introduced
by \citet{Benedek-Itai-1991} but the ideas are much older; see
e.g.~\citet{Dudley-1978, Dudley-1984}. The $O(1/\epsilon)^{O(d)}$ upper bound on
size of the smallest $\epsilon$-cover is by \citet[Lemma 7.13]{Dudley-1978}; see
also \citet[Chapter 4]{Devroye-Lugosi-2000} and \citet{Haussler-1995}.

For any distribution-independent algorithm and any class $C$ of
Vapnik-Chervonenkis dimension $d \ge 2$ and any $\epsilon \in (0,1)$ and $\delta
\in (0,1)$, there exists a distribution over the domain and a concept which
requires at least $\Omega \left(\frac{d + \log(1/\delta)}{\epsilon}\right)$
labeled examples to $\epsilon$-learn with probability at least $1 - \delta$;
see~\citet[Theorem 5.3]{Anthony-Bartlett-1999} and
\citet{Blumer-Ehrenfeucht-Haussler-Warmuth-1989,
Ehrenfeucht-Haussler-Kearns-Valiant-1989}. The proof of the lower bound
constructs a distribution that does \emph{not} depend on the algorithm. The
distribution is a particular distribution over a fixed set shattered by $C$. So
even an algorithm that knows the distribution requires $\Omega \left(\frac{d +
\log(1/\delta)}{\epsilon}\right)$ labeled examples.
